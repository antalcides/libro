\documentclass{scrbook}
\usepackage[ngerman]{babel}
\usepackage[T1]{fontenc}
\usepackage{lmodern}
\usepackage[svgnames]{xcolor}
\usepackage{chngcntr}
\usepackage{amssymb}
\usepackage{aliascnt}
\usepackage{tikz}
\usepackage{lipsum}
\usetikzlibrary{shapes.misc,calc}
\usepackage{tcolorbox}
\tcbuselibrary{skins,theorems,breakable}
\renewcommand*\familydefault{\sfdefault}

\definecolor{hellgrau}{RGB}{215,215,215}
\definecolor{blau}{RGB}{17,94,140}
\definecolor{orange}{RGB}{229,94,30}

\makeatletter
\tcbset{
  mytheorem/.code args={#1#2#3#4}{
    \refstepcounter{#2}\label{#4}
    \pgfkeysalso{title={\setbox\z@=\hbox{#1~\csname the#2\endcsname\ }\hangindent\wd\z@\hangafter=1 \mbox{#1~\csname the#2\endcsname\ }(#3)}}},
}

\newcommand{\mtcbmaketheorem}[5]{
  \newtcolorbox{#1}[3][]{#3,mytheorem={#2}{#4}{##2}{#5:##3},##1}
}
\makeatother

\newcounter{defi}
\newaliascnt{lemm}{defi}
\counterwithin{defi}{chapter}
\counterwithin{lemm}{chapter}

\tcbset{
defstyle/.style={
    breakable,
    freelance,
    boxrule=1pt,
    width=\linewidth,
    frame code={%
    \path[fill=blau]
        ([yshift=-7.5pt]frame.north west) -- ([xshift=7.5pt]frame.north west) --
        (frame.north east) -- (frame.north east|-interior.north east) --
        (frame.north west|-interior.north west) -- cycle;
    },
    interior titled code={
    \path[fill=hellgrau]
        (frame.west|-interior.north west) -- (frame.east|-interior.north east) --   
        (frame.east|-interior.south east) -- (frame.west|-interior.south west) -- cycle;
    \path[draw=white, line width=1pt] ([xshift=-1pt]frame.west|-interior.north west) -- ([xshift=1pt]frame.east|-interior.north east);
    },
    fonttitle=\bfseries\sffamily
},
satzstyle/.style={
    breakable,
    freelance,
    boxrule=1pt,
    width=\linewidth,
    frame code={%
    \path[fill=orange]
        ([yshift=-7.5pt]frame.north west) -- ([xshift=7.5pt]frame.north west) --
        (frame.north east) -- (frame.north east|-interior.north east) --
        (frame.north west|-interior.north west) -- cycle;
    },
    interior titled code={
    \path[fill=hellgrau]
        (frame.west|-interior.north west) -- (frame.east|-interior.north east) --   
        (frame.east|-interior.south east) -- (frame.west|-interior.south west) -- cycle;
    \path[draw=white, line width=1pt] ([xshift=-1pt]frame.west|-interior.north west) -- ([xshift=1pt]frame.east|-interior.north east);
    },
    fonttitle=\bfseries\sffamily
}
}

\mtcbmaketheorem{defi}{Definition}{defstyle}{defi}{df}
\mtcbmaketheorem{lemm}{Lemma}{satzstyle}{lemm}{lm}

\begin{document}

\chapter{A test chapter}

\begin{defi}{Partially ordered set}{poset}
A partial order is a binary relation $\preccurlyeq$ over a set $P$ which is antisymmetric, transitive, and reflexive. A set with a partial order is called a partially ordered set (also called a poset). 
\end{defi}

\begin{lemm}{Zorn's Lemma}{zorn}
Suppose a non-empty partially ordered set $P$ has the property that every non-empty chain has an upper bound in $P$. Then the set $P$ contains at least one maximal element.
\end{lemm}
\ref{item1:satz:PoissPunktProzess}
\lipsum[1-2]

\begin{lemm}{Poissonpunktprozess}{PoissPunktProzess}
Wir nehmen an, unser System zuf"alliger Punkte erf"ullt folgende Bedingungen:
\begin{enumerate}
\item $N_{a,b}$ und $N_{c,d}$ sind stochastisch unabh"angig und $[a,b] \cap [c,d] = \emptyset$. \label{item1:satz:PoissPunktProzess}
\item $N_{a+s,b+s}$ und $N_{a,b}$ haben f"ur alle $s \in [0,\infty)$ die gleiche Verteilung. \label{item2:satz:PoissPunktProzess}
\item Es existiert ein $\lambda > 0$, so dass $\lim_{\Delta t \downarrow 0} \frac{P_1(\Delta t)}{\Delta t} = \lambda$. \label{item3:satz:PoissPunktProzess}
\item Es ist $\lim_{\Delta t \downarrow 0} \frac{P(N_{\Delta t} \geq 2)}{\Delta t} = 0$. \label{item4:satz:PoissPunktProzess}
\end{enumerate}
Dann gilt f"ur $t \geq 0$ bzw. $a,b \in [0,\infty)$, $b > a \geq 0$:
\begin{itemize}
\item $N_t$ ist Poissonverteilt zum Parameter $\lambda t$,
\item $N_{a,b}$ ist Poissonverteilt zum Parameter $\lambda(b-a)$.
\end{itemize}
\end{lemm}

\end{document}