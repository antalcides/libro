\capitulo{Introducci\'on metodol\'ogica e hist\'orica }

%\PartialToc

%\hypersetup{linkcolor=ptctitle}

\vspace*{-1.5cm}
 
\begin{flushright}
\textit{\footnotesize{}El principio de razón suficiente, que afirma
que nada}
\par\end{flushright}{\footnotesize \par}

\begin{flushright}
\textit{\footnotesize{}sucede gratuitamente, es decir, que a todo
fenómeno }
\par\end{flushright}{\footnotesize \par}

\begin{flushright}
\textit{\footnotesize{}le corresponde una explicación, una razón de
ser }
\par\end{flushright}{\footnotesize \par}

\begin{flushright}
\textit{\footnotesize{}que se presente admisible a la razón. }
\par\end{flushright}{\footnotesize \par}

\begin{flushright}
Leibniz: 
\par\end{flushright}

\vspace*{10mm}

Antes de empezar a definir los conceptos abstractos (generales) sobre
los cuales se basa el \textsl{Cálculo} \textsl{Infinitesimal}, nos
parece conveniente desde el punto de vista psicológico, adelantar
algunas ideas acerca de ciertos problemas que dieron origen a esta
rama importante de la \textsl{Matemática Moderna}. el desarrollo lógico
del tema no requiere de esta introducción panorámica, sino que se
bastaría con dar las definiciones abstractas necesarias y después
ir demostrando las relaciones importantes entre los conceptos definidos.
Para la mayoría de los matemáticos este es el único camino <<correcto>>
y esta es la causa de que casi todos los textos de la materia, impecable
la mayoría de ellos desde el punto de vista lógico resulten inasequibles
a los principiantes; y es natural que así sea, pues no todo el mundo
tiene sentido de la adivinación para descubrir de antemano que problemas
puede resolver, qué uso puede dar, a aquellos instrumentos abstractos
que el matemático le va presentando como por arte de magia.

Los que proceden de esta forma, es decir limitándose a dar las definiciones
y teoremas, sin anticipar una descripción global ( aunque no sea definitivamente
rigurosa) de los problemas que requieren para su solución aquellas
definiciones y teoremas , condenan al alumno al hastío y la esterilidad,
pues dejan inoperantes las fuerzas vitales concurrentes en el proceso
de enseñanza-aprendizaje. No hay que olvidar que <<El martillear
fue antes que el martillo>>; los problemas fueron anteriores a los
métodos para resolverlos, y es difícil, para la mayoría de las inteligencias,
moverse comprensivamente, en un plano de ideas abstractas sin entrever
de donde fueron abstraídas. de aquí la necesidad psicológica ( no
lógica) de anticipar la descripción de los problemas que dieron origen
a los nuevos métodos, pues ellos fueron la base de lo que nació síntesis
de conceptos y relaciones que constituyen el moderno \textsl{Cálculo}
\textsl{Infinitesimal. }Si a ser inteligente, que nunca hubiese visto
un martillo, se le muestra uno, indicándole previamente que ese instrumento
se va a usar para martillear, comprenderá muy bien por qué tiene el
diseño que todos nosotros conocemos, y hasta es posible que, si tiene
genio para ello, idee un nuevo diseño para el martillo, más adecuado
para la acción de martillear. Consecuentemente con estas ideas, hemos
decidido, aunque esto sea salirnos de los moldes euclidianos de la
mayoría de los textos, escribir el libro que nos hubiese gustado leer
cuando teníamos quince o dieciséis años, anticipando la descripción
de los problemas más importantes que dieron origen al Cálculo Infinitesimal. 

Los problemas típicos que dieron origen al Cálculo Infinitesimal comenzaron
a plantearse en la época clásica de Grecia (hacia el siglo III antes
de nuestra era cristiana) y no se encontraron métodos generales y
sistemáticos de resolución hasta veinte siglos después ( en el siglo
XVII por obra de Newton y Leibniz). 

Se debe al matemático y filósofo griego \textsl{Eudoxio} el <<método
de Exhaustación>> (método de agotamiento) para resolver ciertos problemas
típicos de lo que, pasados los siglos, habría de llamarse \textsl{Cálculo
Infinitesimal}. Con dicho método se atacaron con éxito los problemas
relativos al cálculo de las áreas y volúmenes de las figuras elementales
(círculos, esferas, conos, pirámides, etc.)

por la misma época, el más grande de los matemáticos griegos, y uno
de los mayores de toda la historia de la humanidad, \textsl{ARQUIMEDES},
resolvió problemas, por el <<Método de los Recubrimientos>>, tan
difíciles que, aún hoy en día , después de conocer los modernos <<Métodos
Infinitesimales>>, resultan laboriosos. por este motivo se le llama
<<el precursor del Cálculo Infinitesimal>>.

Describamos uno de estos problemas típicos, que fue resuelto con éxito
por Arquímides y que nosotros resolveremos al final de este libro
con los métodos modernos.

\begin{problema}[Segmento de Arquímides]\label{prob:pro1} 

Hallar el área de un segmento de parábola.

\end{problema}

\begin{solucion}

\begin{wrapfigure}{r}{0.4\linewidth} \centering

\includegraphics[scale=0.6]{3_home_antalcides_MEGA_calculo_I_libro_pdf_cal_int1.pdf}
\caption{Segmento de una  Parábola $POP'$.}
\label{fig:myfig1} \end{wrapfigure}

Si por un punto $M$ del eje de una parábola trazamos un segmento
$PP'$perpendicular a dicho eje, en la figura \ref{fig:myfig1} $POP'$
representa la región determinada por el segmento parabólico y el segmento
$PP'$el cual es llamado <<\textsl{Segmento de parábola}>>.

El método seguido por Arquímides para hallar el área del \textsl{Segmento
parabólico} $POP'$ fue el siguiente: recubrir el segmento de parábola
por una serie de rectángulos inscritos con bases paralelas a $PP'$
y alturas muy pequeñas. En la figura \ref{fig:myfig1} se destaca
uno de tales rectángulos inscritos, así como. Si suponemos que la
altura del Segmento parabólico como $h=OM$, se divide por ejemplo
en 100 partes iguales, y por los puntos de división se trazan paralela
a $PP'$ (base del segmento parabólico), se tendrían 100 rectángulos
inscritos y otros 100 circunscritos.

Se dice que los 100 rectángulos inscritos constituyen un <<recubrimiento
interno>> del Segmento parabólico y los cien rectángulos circunscritos
un <<recubrimiento externo>>, ahora se comprende fácilmente que:
la suma de las áreas de los rectángulos del recubrimiento interno
será menor que el área del segmento parabólico y la suma de las áreas
de los rectángulos del recubrimiento externo será mayor que el área
de dicho Segmento parabólico.

Supongamos que el área del recubrimiento externo menos la del interno
fuese igual a 0.1 $\si{cm^2}$; en este caso podría tomarse como el
área del Segmento parabólico; bien la del recubrimiento interno o
la del recubrimiento externo, con un error menor de 0.1 $\si{cm^2}$.
Si <<queremos afinar más>> el resultado (es decir reducir el error)
podríamos dividir la altura $OM$ en 1000 partes iguales , en vez
de 100, y obtener lo

s respectivos recubrimientos externos e internos, formados por los
rectángulos circunscritos e inscritos , respectivamente . el área
de cualquiera de estos recubrimientos es una aproximación más precisa
que la proporcionada anteriormente.

Se comprende que si quisiéramos el valor exacto del área del segmento
parabólico, tendríamos que proseguir indefinidamente la operación
de dividir la altura $OM$ en un número cada vez mayor de partes iguales,construyendo
los correspondientes recubrimientos del segmento parabólico con rectángulos.

\end{solucion}

La idea contenida en este método es genial, y como todas las ideas
geniales esta es extraordinariamente simple y directa, surgida de
la esencia misma del problema: Recubrir una magnitud desconocida (el
Segmento parabólico) con piezas conocidas (rectángulos); la novedad
que se presenta es que, el número de piezas necesarias para que el
recubrimiento se confunda con el Segmento parabólico es un número
infinito. (estas ideas descriptivas serán presentadas más adelante
con todo rigor, pues son de gran importancia en la Matemática Superior
Moderna). 

\begin{resumen}\peque

\textsl{En resumen:}

A la operación de sumar las áreas de los rectángulos que <<integran>>
cada recubrimiento hay que añadir una operación nueva, que definiremos
después con precisión, y que se llama <<operación de paso al límite>>,
y que consiste en obtener el valor correspondiente al área de un recubrimiento
cuando el número de sus rectángulos crece indefinidamente.

\end{resumen}

El ejemplo que acabamos de describir corresponde a un problema típico
del \textsl{Cálculo Integral}, y como tal será resuelto después.

Otro de los problemas históricos que dieron origen al \textsl{Cálculo}
\textsl{Infinitesimal }es el problema de la recta tangente , el cual
describiremos a continuación.

\begin{problema}[Pendiente de la recta tangente]\label{prob:pro2} 

Hallar la pendiente de una recta tangente a una curva.

\end{problema}

\begin{solucion}

\begin{wrapfigure}{r}{0.4\linewidth} \centering

\includegraphics[scale=0.75]{4_home_antalcides_MEGA_calculo_I_libro_pdf_cal_int2.pdf}
\caption{Recta tangente a $f(x)$.}
\label{fig:myfig2} \end{wrapfigure}

Sea dada una curva, cuya ecuación referida  a un sistema de ejes cartesianos
cartesianos, donde $y=f(x)$como se muestra en la figura \ref{fig:myfig2}.
Se define la tangente a la curva en uno de sus puntos. $P_{0}$ ,
como la posición límite de las secantes $P_{0}P_{n}$, cuando $P_{n}$
se mueve sobre la curva aproximándose indefinidamente a $P_{0},$
teniendo a confundirse con él. (Estas nociones descriptivas serán
precisadas en el capítulo 1, cuando demos el concepto riguroso de
infinitésimo). 

Si las coordenadas de $P_{0}$ son $(x_{0},y_{0})$, para determinar
la ecuación de la recta $\overleftrightarrow{P_{0}T}$, tangente a
la curva en $P_{0}$, bastará determinar su pendiente, pues según
sabemos por Geometría Analítica, si la recta pasa por $P_{0}(x_{0},y_{0})$,
y tiene pendiente conocida , $m,$ la ecuación de dicha recta es:
$y-y_{0}=m\left(x-x_{0}\right),$ (recta que pasa por $P_{0}\left(x_{0},y_{0}\right)$y
tiene pendiente $m$). El problema de la determinación analítica de
la tangente a una curva en uno de sus puntos se reduce por tanto a
la determinación de la pendiente o coeficiente angular de dicha tangente.
Describamos el proceso mediante el cual se logra esta determinación. 

Sea $P_{1}$ otro punto sobre la curva $f(x)$. Los puntos $P_{0}$
y $P_{1}$ determinan una recta secante $\overleftrightarrow{P_{1}P_{0}}$
a $f\left(x\right)$ , y si $P_{0}$ está muy próximo a $P_{1}$la
secante $\overleftrightarrow{P_{1}P_{0}}$ estará muy próxima a $\overleftrightarrow{P_{0}T}$
en $P_{0}$.

Tratando por $P_{0}$una paralela al eje $OX$, y por $P_{1}$ una
paralela al eje $OY$, se forma el triángulo $P_{0}P_{1}P'_{1},$
que llamaremos triángulo de incrementos, pues el cateto $P_{o}P'_{1}$
es el incremento que experimenta la $x$ al pasar de $P_{0}$a $P_{1}$,
y el cateto $P'_{1}P_{1}$es el incremento que experimenta la $y$,
al pasar de $P_{0}$ a $P_{1}$. Representaremos estos incrementos
así: $P_{0}P'_{1}=\Delta_{1}x_{0}$ y $P'_{1}P_{1}=\Delta y_{0}$
respectivamente. El cociente que resulta de dividir el incremento
de la ordenada por el de la abscisa, o sea: $\dfrac{\Delta_{1}y_{0}}{\Delta_{1}x_{0}},$
recibe el nombre de cociente incremental y será muy usado en este
libro.

Si llamamos $\alpha_{1}$ al ángulo $P'_{1}P_{0}P_{1},$ se ve en
la figura \ref{fig:myfig2} que 
\begin{equation}
\tan\alpha_{1}=\dfrac{\Delta_{1}y_{0}}{\Delta_{1}x_{0}}
\end{equation}

Como hemos dicho antes, la $\tan\alpha_{1}$ es un valor aproximado
de la pendiente de la tangente, $P_{0}T,$ a la curva $P_{0}.$

Si quisiéramos aumentar la precisión en el valor aproximado de la
$\tan\alpha_{1}$, tomaríamos un nuevo punto, $P_{2},$ sobre la curva,
más próximo a $P_{0}$ que el punto $P_{1}.$ Construiríamos la secante
$\overleftrightarrow{P_{0}P_{2}}$ y trazaríamos por $P_{2}$ la paralela
al eje $OY,$ con lo que se formaría el nuevo triángulos de incrementos
$P_{0}P'_{2}P_{2}$(con ángulo recto en $p'_{2}$). Si llamamos $\alpha_{2}$al
ángulo $P'_{2}P_{p}P_{2}$se tendría:
\begin{equation}
\tan\alpha_{2}=\dfrac{\Delta_{2}y_{0}}{\Delta_{2}x_{0}},\text{(véase la figura \ref{fig:myfig2}) }
\end{equation}

Si quisiéramos aproximar aún más la pendiente, tomaríamos un nuevo
punto sobre la curva $f(x)$, $P_{0}$que los anteriores, y con construcciones
análogas a las descritas para los puntos $P_{1}$ y $P_{2},$y con
la notación también análoga, se tendría:
\begin{equation}
\tan\alpha_{3}=\dfrac{\Delta_{3}y_{0}}{\Delta_{2}x_{0}},
\end{equation}

que sería un valor más aproximado de la pendiente de la recta tangente
$\overleftrightarrow{P_{0}T}$,

Si continuáramos indefinidamente el proceso de acercamiento de los
puntos sobre la curva $f\left(x\right)$al punto $P_{0},$ obtendríamos
la siguiente sucesión de coeficientes incrementales: 

\begin{equation}
\tan\alpha_{1}=\dfrac{\Delta_{1}y_{0}}{\Delta_{1}x_{0}},\,\tan\alpha_{2}=\dfrac{\Delta_{2}y_{0}}{\Delta_{2}x_{0}},\,\tan\alpha_{3}=\dfrac{\Delta_{3}y_{0}}{\Delta_{3}x_{0}},\cdots,\tan\alpha_{n}=\dfrac{\Delta_{n}y_{0}}{\Delta_{n}x_{0}}\label{eq:ec4}
\end{equation}

Cuán más grande sea el valor de $n$ en la ecuación \eqref{eq:ec4},
más próximo será el valor de $\alpha_{n}$ al valor de la pendiente
de $\overleftrightarrow{P_{0}T}$ en $P_{0}$; pero si queremos el
valor exacto, no aproximado, de la pendiente de la tangente, tendremos
que calcular el valor hacia el cual tienden los cocientes incrementales
dados el la ecuación \eqref{eq:ec4}, cuando el punto $P_{n}$ tiende
a confundirse con el punto $P_{0};$ es decir, nos encontramos nuevamente
con la necesidad de efectuar la nueva operación que se será estudiada
en el capítulo 1 de este libro y que se llama <<\textsl{operación
de paso al límite}>>.

\end{solucion}

Los límites del tipo que aquí descrito reciben el nombre de derivadas.
El método seguido en la descripción de este problema permite descubrir
una analogía con el problema del área del \textsl{Segmento parabólico}:
Se calcula una sucesión de aproximaciones, cada vez más finas, de
la magnitud que se desea obtener; después, por una operación nueva
, de <<\textsl{paso al límite}>>, efectuada sobre estas aproximaciones,
se obtiene el valor exacto de la magnitud incógnita. Esta analogía
se mantiene en los problemas que describiremos a continuación , y
en todos los problemas del \textsl{Cálculo Infinitesimal}; de ahí
la importancia de realizar un estudio sistemático y general de las
sucesiones <<\textsl{sucesiones de números reales}>>, (como la de
recubrimientos en el problema (\ref{prob:pro1}), la de los cocientes
incrementales, en el problema (\ref{prob:pro2}), y otras muchas más
que nos encontraremos a lo largo del libro), y de la forma de obtener
sus límites. Puede decirse que toda la tarea del \textsl{Cálculo Infinitesimal}
consiste en dar reglas para obtener ciertos límites especiales que
ocurren con bastante frecuencia en la solución de problemas importantes.

\begin{problema}[Velocidad Instantánea]\label{prob:pro3} 

Hallar la velocidad instantánea de un movimiento variado.

\end{problema}

\begin{solucion}

Sabemos de la Física, que un movimiento se llama uniforme cuando un
móvil recorre espacios iguales en intervalos de tiempos iguales, y
variado cuando los espacios recorridos en intervalos iguales no son
iguales. 

En el caso del movimiento uniforme los espacios son proporcionales
a los intervalos de tiempos invertidos en recorrerlos, pues a doble,
triple, ..., intervalos de tiempo, corresponde el doble, triple,...,
espacio.

La constante de proporcionalidad recibe el nombre de velocidad del
movimiento y es igual al cociente de dividir el espacio recorrido
entre el intervalo de tiempo empleado.

Si representamos al espacio recorrido por el móvil por $e$, y el
tiempo empleado por $t$, y a la velocidad por $v$, se cumple para
el movimiento uniforme que $v=\dfrac{e}{t}$ es constante, siendo
el espacio $e$ el espacio recorrido y $t$ el tiempo empleado cantidades
con las unidades correspondientes a un sistema de medidas cualquiera.

La característica esencial del movimiento del movimiento uniforme
es la constancia de la velocidad a lo largo del recorrido; pero en
un movimiento variado la velocidad cambia de unos puntos a otros del
recorrido.

Si queremos tener un conocimiento exacto de un movimiento variado
necesitamos conocer la velocidad del móvil en cada punto del trayecto,
es decir, la velocidad en cada instante del tiempo que dura el movimiento,
que recibe el nombre de \textsl{<<velocidad instantánea}>>.

Para simplificar la exposición del método seguido para obtener la
<<\textsl{velocidad instantánea}>> de un movimiento variado, supondremos
que el móvil se mueve sobre la recta $AB$, en negativo, como se muestra
en la figura (\ref{fig:fig3}).

\begin{figure}[H]
\centering\includegraphics{5_home_antalcides_MEGA_calculo_I_libro_pdf_cal_int3.pdf}\caption{Velocidad instantánea}\label{fig:fig3}
\end{figure}

Sea $P$ el punto del trayecto en el cual queremos obtener la velocidad
del móvil; el espacio lo tomaremos como la longitud del segmento $PP_{1},$
y lo representaremos como $\Delta_{1}e$ ( primer incremento de espacio)
y el tiempo invertido en recorrer $\Delta_{1}e$ lo representaremos
por $\Delta_{1}t$. Emplearemos una notación análoga para los espacios
$PP_{2},\;PP_{3},\cdots,PP_{n}$, y para sus tiempos correspondientes. 

Describamos como se obtendría la velocidad en el punto $P$. El cociente
incremental, $\dfrac{\Delta_{1}e}{\Delta_{1}t}=\dfrac{PP_{1}}{\Delta_{1}t}\inline\text{\label{eq:ec5}}$\foreignlanguage{english}{,}
nos da la velocidad promedio en el trayecto $PP_{1}$, y si dicho
trayecto es pequeño, cociente incremental (\ref{eq:ec5}) es un valor
aproximado de la velocidad en el punto $P$. Si queremos aproximarnos
más al valor de la velocidad en el punto $P$ consideraríamos un nuevo
punto, $P_{2}$ más próximo a $P$ que $P_{1},$ y hallaríamos $\dfrac{\Delta_{2}e}{\Delta_{2}t}=\dfrac{PP_{2}}{\Delta_{2}t},$
que es la velocidad promedio en el trayecto $PP_{2},$ más próxima
que la anterior a la velocidad en el punto $P.$ 

Si queremos acercarnos aún más al valor de la velocidad en el punto
$P,$ consideraríamos un nuevo punto, $P_{3},$ más cercano a $P$
que los anteriores, obteniendo; $\dfrac{\Delta_{3}e}{\Delta_{3}t}=\dfrac{PP_{3}}{\Delta_{3}t}.$

Prosiguiendo indefinidamente este proceso de forma que los cocientes
incrementales de los espacios recorridos para alcanzar $P$ divididos
por los tiempos empleados en recorrerlos, para espacios cada vez más
pequeños, se obtiene la siguiente sucesión de cocientes incrementales:
\begin{equation}
\dfrac{\Delta_{1}e}{\Delta_{1}t},\;\dfrac{\Delta_{2}e}{\Delta_{2}t},\;\dfrac{\Delta_{3}e}{\Delta_{3}t},\cdots,\dfrac{\Delta_{n}e}{\Delta_{n}t}\cdots.\label{eq:ec6}
\end{equation}

Los valores de los términos de la sucesión dada por la ecuación \eqref{eq:ec6}
se aproximan más y más a la velocidad del movimiento en el punto $P,$
pero si queremos el valor exacto de dicha velocidad, necesitamos de
la nueva operación, ya citada en los ejemplos anteriores: <<\textsl{La
operación de paso al límite}>> que consiste en hallar el valor hacia
el cual tienden los términos de la sucesión dada en \eqref{eq:ec6}
cuando $P_{n}$ tiende a $P,$ de forma que $\Delta_{n}t$ tienda
a cero.

\end{solucion}

Este problema que acabamos de describir, es de importancia capital
en la Física, necesita también para ser resuelto del instrumento da
la <<teoría de los límites>>. este problema quedará clasificados
dentro de los límites que llamaremos \textsl{derivada}.

El lector habrá captado la identidad estructural entre el problema
(\ref{prob:pro3}) y el de la tangente (problema: \ref{prob:pro2}).
Ambos son problemas que conducen a límites de cocientes incrementales
que reciben al aplicarle << \textsl{el paso al límite}>> el nombre
de derivada. 

\begin{problema}[Trabajo realizado por una fuerza]\label{prob:pro4} 

Hallar el trabajo realizado por una fuerza variable a lo largo de
un segmento de linea recta.

\end{problema}

\begin{solucion}
\begin{enumerate}
\item[a)]  Si una \textsl{fuerza constante} actúa sobre un cuerpo a lo largo
de un \textsl{segmento rectilíneo}, $AB$, de longitud $l,$ tenemos
de acuerdo con la \textsl{Física} que el \textsl{trabajo} realizado
por dicha \textsl{fuerza constante} sobre el objeto se obtiene del
producto de la magnitud de la fuerza $F$ por el desplazamiento $l.$
\item[b)]  Si una \textsl{fuerza variable} actúa sobre un cuerpo a lo largo
de un \textsl{segmento rectilíneo}, $AB,$ de longitud $l,$ el cálculo
del \textsl{trabajo} realizado por esta fuerza sobre un cuerpo se
complica extraordinariamente, hasta el extremo de necesitarse una
\textsl{operación de paso al límite} como veremos a continuación.
\end{enumerate}
\begin{figure}[H]
\centering\includegraphics[scale=0.8]{6_home_antalcides_MEGA_calculo_I_libro_pdf_cal_int4.pdf}\caption{Trabajo realizado por una fuerza}\label{fig:fig4}
\end{figure}

\vspace*{2pt}\begin{nota}\peque

\emph{\noun{Descripción}}: si dividimos el segmento $AB$ en un número
de partes iguales, por ejemplo 100 partes iguales (en la figura (\ref{fig:fig4})
se señala una parte, de centro en $x$ y una fuerza horizontal aplicada
en al centro del objeto, igual a $f\left(x\right),$podemos considerar
que la fuerza horizontal en cada parte es constante (lo cual no es
cierto, pues hemos supuesto que la fuerza es continuamente variable
a lo largo de $AB$, <<en realidad $f\left(\Delta x\right)$ es aproximadamente
constante en cada $\Delta x=\dfrac{AB}{100},$ si es bien pequeño>>),
e igual valor que corresponde a la fuerza horizontal aplicada en el
centro del objeto, es decir para cada $\Delta_{i}x$ se tiene que
$f\left(\Delta_{i}x\right)=f\left(x_{i}\right)$ para todo $i=1,2,3,\cdots,100$.

\end{nota}\vspace*{2pt} 

Con esta simplificación podemos calcular el trabajo realizado por
la fuerza horizontal sobre el objeto en cada parte, al cual llamaremos
\textsl{trabajo elemental} o \textsl{elemento de trabajo}: el trabajo
de cada parte es igual al producto del recorrido, $\Delta_{i}x$,
por la fuerza horizontal sobre el objeto, $f\left(\Delta_{i}x\right).$
se comprende fácilmente que si sumamos los elementos de trabajo en
las 100 partes obtendremos un valor aproximado del trabajo total realizado
por la fuerza horizontal variable a lo largo del segmento rectilíneo
$AB.$ 

Si queremos afinar más la aproximación dividimos el segmento $AB$
en 1000 partes iguales y procederíamos igual que antes, calculando
el trabajo elemental de cada parte como el producto de su desplazamiento
por la fuerza horizontal y sumando los mil trabajos elementales, es
decir 
\[
W_{Total}\approx f\left(\Delta_{1}x\right)\Delta_{1}x+f\left(\Delta_{2}x\right)\Delta_{2}x+f\left(\Delta_{3}x\right)\Delta_{3}x+\cdots+f\left(\Delta_{1000}x\right)\Delta_{1000}x.
\]
Podríamos tener obtener valores más aproximados si dividimos $AB$
por un numero cada vez mayor de partes iguales y calculamos el trabajo
sobre cada parte y luego sumamos todos los trabajos, pero si queremos
el valor exacto del trabajo realizado por la fuerza horizontal variable
para desplazar el objeto sobre el segmento rectilíneo $AB$, debemos
proseguir indefinidamente la división del segmento rectilíneo $AB$
en segmentos cada vez más pequeños y pasar al límite las sumas de
los trabajos elementales cuando el número de partes iguales en las
que se divide el segmento $AB$ tiende a infinito.

\end{solucion}

Este problema tiene una estructura matemática similar al problema
(\ref{prob:pro1}); ambos conducen, como veremos más adelante, a un
tipo de límite que recibe el nombre de\textsl{ integral definida}. 

\begin{problema}[Cuerpo con densidad heterogénea]\label{prob:pro} 

Hallar la masa de un cuerpo con densidad heterogénea.

\end{problema}

\begin{solucion}

Una sustancia es homogénea si cuando volúmenes iguales del mismo material
tienen la misma masa, es decir, que una fuerza de igual magnitud produce
sobre dos cuerpos de igual volumen y de la misma sustancia aceleraciones
diferentes.

Cuando no se cumple la condición anterior se dice que la sustancia
es heterogénea.
\begin{enumerate}
\item[a)]  Si una sustancia es homogénea, el cociente de dividir su masa por
su volumen es constante, cualquiera que sea el volumen de la sustancia
que se se considere, y este cociente recibe el nombre de \textsl{densidad}.
La densidad de una sustancia homogénea se mide en gramos por centímetro
cúbico. ($\si{g/cm^3}$).(Un gramo es la masa que corresponde 1\si{cm^3}
de agua pura a la temperatura de $4^{\circ}$ centígrados).
\item[b)]  Sustancia heterogénea 
\end{enumerate}
Consideremos una sustancia, que para simplificar la explicación la
supondremos en un recipiente cilíndrico de sección de $1\si{cm^2}$,(como
muestra la figura(\ref{fig:myfig5}). Supondremos además que la sustancia
es heterogénea y que su densidad cambia de forma continua al variar
la altura, $x,$ de las capas de sustancia al fondo del recipiente.
Se supone finalmente que todos los puntos de un plano paralelo al
del fondo (de igual altura sobre el fondo) tienen la misma densidad.

¿Cómo haríamos para calcular la densidad de la sustancia en la capa
situada a una altura $x$ sobre el fondo?

\vspace*{2pt}\begin{nota}\peque

\emph{\noun{Descripción}}: Se rodea a la capa de altura $x$ de un
volumen comprendido entre dos capas simétricas de la anterior separadas
una distancia igual a $\Delta_{1}x$ ; puesto que hemos considerado
la sección del recipiente igual a $1\si{cm^2}$, este volumen coincidirá
con la capa de altura $\Delta_{1}x$ (vea la figura: \ref{fig:myfig5}) 

\end{nota}\vspace*{2pt} 

\begin{wrapfigure}{r}{0.4\linewidth} \centering

\includegraphics[scale=0.55]{7_home_antalcides_MEGA_calculo_I_libro_pdf_cal_int5.pdf}
\caption{densidad de una sustancia heterogénea.}
\label{fig:myfig5} \end{wrapfigure}

Si dividimos la masa que hay entre las dos capas que rodean a la capa
de altura $x,$ que representamos por $\Delta_{1}m,$ por el volumen
comprendido entre ellas, que ya hemos dicho que es $\Delta_{1}x,$
obtenemos $\rho_{1}=\dfrac{\Delta_{1}m}{\Delta_{1}x}\inline\text{\label{eq:ec7}}$.

El cociente incremental de la ecuación (\ref{eq:ec7}) es un valor
aproximado de la densidad de la capa de altura $x,$ tanto más aproximado
cuanto menor sea $\Delta_{1}x;$ él representa la \textsl{densidad
media} de las capas comprendidas en el volumen considerado.

Si queremos aproximarnos más al valor exacto de la densidad correspondiente
a la capa de altura $x,$ trazamos dos nuevas capas simétricas también
de la altura $x,$y más próximas a ella que las anteriores, separadas
entre sí una distancia $\Delta_{2}x<\Delta_{1}x.$ Con notación análoga
a la anterior, se tiene: $\rho_{2}=\dfrac{\Delta_{2}m}{\Delta_{2}x}\inline\text{\label{eq:ec8}.}$

Este proceso de aproximaciones sucesivas hacia la densidad verdadera
de la capa de altura $x$ se puede proseguir indefinidamente, obteniendo
la siguiente sucesión de densidades medias, cada vez más próximas
a la densidad buscada:
\begin{equation}
\rho_{1}=\dfrac{\Delta_{1}m}{\Delta_{1}x},\;\rho_{2}=\dfrac{\Delta_{2}m}{\Delta_{2}x},\;\rho_{3}=\dfrac{\Delta_{3}m}{\Delta_{3}x},\cdots,\rho_{n}=\dfrac{\Delta_{n}m}{\Delta_{n}x}.\label{eq:ec9}
\end{equation}

Pero si queremos obtener el valor exacto de la densidad de la capa
de altura $x,$ necesitamos también en esta ocasión efectuar una operación
de\textsl{ paso al límite }sobre la sucesión \ref{eq:ec9}, o sea,
encontrar el valor hacia el cual tiende la sucesión de densidades
medias cuando $\Delta_{n}x$ tiende a cero. Este problema corresponde
a un límite análogo a los que se presentaron en los problemas \ref{prob:pro2}
y \ref{prob:pro3}, que después será llamado derivada.

\end{solucion}

\vspace*{2pt}\begin{nota}\peque

\emph{\noun{Conclusión}}: El lector que haya reflexionado atentamente
sobre las descripciones de los problemas anteriores, comprenderá la
necesidad de un estudio sistemático y general de esta operación que
le hemos hecho entrever y que hemos denominado de \textsl{paso al
límite}; sin ella, todos estos problemas, y otros muchísimos análogos
de capital importancia en Física, Química, Estadística, Economía,
etc., quedarían sin resolver.

\end{nota}\vspace*{2pt} 
