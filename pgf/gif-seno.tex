\documentclass[tikz,dvipsnames,svgnames]{standalone}
\usepackage[utf8]{inputenc} 
\usepackage[upright]{fourier}
\usepackage{xcolor}
%\usepackage{xcolor,color}
\usepackage{sansmath}
\usepackage{tikz}
%\usepackage{pgfplots}
\usepackage{siunitx}
\usepackage{fix-cm}
\usepackage{amsmath,tkz-fct}
\usepackage{tkz-tab,tkz-euclide}
\usetkzobj{all}
%%%%%%%%%%%%%% definición de colores %%%%%%%%%%%
\definecolor{bistre}{rgb}{.75,.50,.30}
\definecolor{Maroon}{rgb}{0.5,0.0,0.0}%{.80,.80,.95}
\definecolor{fondpaille}{cmyk}{0,0,0.1,0}
\definecolor{ocre}{RGB}{243,102,25}
\newcommand\BoxColor{ocre!60}
\definecolor{mygreen}{rgb}{0.61,0.73,0.35}
\definecolor{miblue}{RGB}{0,163,243}
\definecolor{captionbgcolor}{RGB}{103,143,150}
\definecolor{paper}{RGB}{239,227,157}
\definecolor{titlebgdark}{RGB}{0,163,243}
\definecolor{titlebglight}{RGB}{191,233,251}
\definecolor{secnum}{RGB}{13,151,225}
\definecolor{ptcbackground}{RGB}{212,237,252}
\definecolor{ptctitle}{RGB}{0,177,235}
\definecolor{myred}{RGB}{0,163,243}
\definecolor{myyellow}{RGB}{0,177,235}
\definecolor{zanahoria}{RGB}{235,59,0}
\definecolor{problemblue}{RGB}{100,134,158}
\definecolor{idiomsgreen}{RGB}{0,162,0}
\definecolor{exercisebgblue}{RGB}{192,232,252}
\definecolor{mygray}{RGB}{215,215,215}
\definecolor{myblue}{RGB}{17,94,140}
\definecolor{gray97}{gray}{.97}
\definecolor{mauve}{rgb}{0.58,0,0.82}
\definecolor{dkgreen}{rgb}{0,0.6,0}
%%%%%%%%%%%%%%%%%%%%%%%%%%%%%%%%%%%%%%%%%%%%%%%%%
\pagecolor{fondpaille}
\color{Maroon}
\tkzSetUpColors[background=fondpaille,text=Maroon]
%\usetikzlibrary{shadings,intersections}
%\usetikzlibrary{calc,patterns,angles,quotes}
%\usetikzlibrary{shapes.emoticon}
\usetikzlibrary{bending,arrows,automata,chains,positioning}
%\usetikzlibrary{arrows.meta}
%\usetikzlibrary{graphs,graphs.standard,quotes}
\usepackage{relsize}
%\usepackage[active,pdftex,tightpage]{preview}
\usepackage{verbatim}
\usepackage[upright]{fourier}
\def\sen{\mathop{\mbox{\normalfont sen}}\nolimits}
\begin{document}
\foreach \angle in {0,10,...,360}
{
  \begin{tikzpicture}[
                 > = stealth',
         shorten > = 1pt,
     node distance = 13mm and 0mm,
             auto,
      start chain = 1 going below,
      start chain = 2 going right,
every label/.style = {font=\tiny, align=left},
every state/.style = {draw=blue!50,very thick,fill=blue!20,on chain=1},
       X/.style = {rectangle, on chain=2}
                        ]
    % Coloreando el circulo y la funci\'on
    \fill[ptctitle] (-1,0) arc (0:\angle:1) -- (-2,0) -- cycle;
    \fill[ptctitle] plot[smooth,domain=0:\angle] (pi/180*\x,{sin(\x)}) |- (0,0);
      % dibujando los ejes con sus escalas
    \draw (-3.5,0) -- (7,0);
    \foreach \deg in {90, 180, 270, 360}
      \draw (pi/180*\deg,2pt) -- (pi/180*\deg,-2pt) node[below] {$\deg^\circ$};
    \draw (0,-1.2) -- (0,1.8);
    \foreach \y in {-1,-0.5,0.5,1}
      \draw(2pt,\y) -- (-2pt,\y) node[left] {$\y$};
    % Dibujando el circulo generatriz y la función seno
    \draw[titlebgdark] plot[smooth,domain=0:360] (pi/180*\x,{sin(\x)});
    \draw[titlebgdark] (-2,0) circle (1);
    \draw[ocre,thick] (pi/180*\angle,0) --(pi/180*\angle,{sin(\angle)}) ;
    \draw[ocre,thick] ({-2+cos(\angle)},0) --({-2+cos(\angle)},{sin(\angle)}) ;
    % trasladando el punto del circulo a la funci\'on
    \draw[fill=zanahoria,zanahoria,dashed] (-2,0) +(\angle:1) circle (1pt) -- (pi/180*\angle,{sin(\angle)}) circle (1pt);
    %%%%%%%%%
    \coordinate (D) at (0, 1.45);
    %\draw(2,0.75)\node (D) [state] {};
    \node (d1) [X,fill=blue!30,right=11mm of D]     {Funci\'on Seno};
    \node (d2) [X,fill=green!30,right=of d1.east]         {$\sen({\pgfmathparse{\angle}\pgfmathresult}^\circ)=\pgfmathparse{sin(\angle)}\pgfmathresult$};
      \end{tikzpicture}
}
\end{document}
%orden en consola
convert -density 300 -delay 8 -loop 0 -background white -alpha remove gif-seno.pdf gif-seno.gif

