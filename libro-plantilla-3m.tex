\batchmode
\makeatletter
\def\input@path{{style/}{section/}}
\makeatother
\documentclass[3m,twoside]{libro-matua}
\usepackage[utf8x]{inputenc}
\usepackage[spanish]{babel}
\usepackage{makeidx}
\usepackage{graphicx}
\usepackage{fancyhdr}
\graphicspath{{ps/}{logo/}{image/}{sections/Figures/}}
%\usepackage{lmodern}
\usepackage{lipsum}
\usepackage[hmargin=2cm,bmargin=3cm,tmargin=4.5cm,centering]{geometry}
%%%%%%%%%%%%%%%%%%%%%%%%%%%%%%%%%%%Encabezados y pie de p\'agina %%%%%%%%%%%%%%%%%%%%
\pagestyle{fancy}
\fancyhf{}
%%%%%%%%%%%%%%%%%%%%encabezado%%%%%%%%%%%%%%%%%%%%%%%%%%%%%%%
\fancyhead[LE]{\hspace*{-0.2\headwidth}\colorbox{ptctitle!20}{\makebox[\dimexpr0.2\headwidth-2\fboxsep][c]{\strut\bf\color{ptctitle}\thepage}}%
               \colorbox{zanahoria!60}{\makebox[\dimexpr\headwidth-2\fboxsep][l]{\strut\bf\color{white}\sffamily\itshape\small\protect\nouppercase{\rightmark}}}}
 %%%%%%%%%%%%%%%%%%%%%%%%%%%              
\fancyhead[LO]{\colorbox{zanahoria!60}{\makebox[\dimexpr\headwidth-2\fboxsep][r]{\strut\bf\color{white}\sffamily\itshape\small\protect\nouppercase{\rightmark}}}%
                \colorbox{ptctitle!20}{\makebox[\dimexpr0.2\headwidth-2\fboxsep][c]{\strut\bf\color{ptctitle}\thepage}}}
 %%%%%%%%%%%%%%%%%%%%%%%%%%%%%%%%%% pies de p\'agina %%%%%%%%%%%%%%%%
 \fancyfoot[LE]{\hspace*{-0.2\headwidth}\colorbox{ptctitle!20}{\makebox[\dimexpr 0.6\headwidth-2\fboxsep][r]{\strut\bf\color{ptctitle} Cálculo Diferencial }}%
               \colorbox{zanahoria!60}{\makebox[\dimexpr 0.6\headwidth-2\fboxsep][l]{\strut\bf\color{white}\sffamily\itshape\small\protect\nouppercase{Autor1\hfill Autor2 \hfill Autor3}}}}               
               %%%%%%%%%%%%%%%%%%%%%%
                \fancyfoot[LO]{\hspace*{-0.2\headwidth}\colorbox{ptctitle!20}{\makebox[\dimexpr0.6\headwidth-2\fboxsep][c]{\strut\bf\color{ptctitle}Cálculo Diferencial}}%
               \colorbox{zanahoria!60}{\makebox[\dimexpr0.6\headwidth-2\fboxsep][r]{\strut\bf\color{white}\sffamily\itshape\small\protect\nouppercase{Autor1\hfill Autor2 \hfill Autor3}}}}               
%%%%%%%%%%%%%%%%%               
\renewcommand{\headrulewidth}{0pt}
%%%%%%%%%%%%%%%%%%%%%%%%%%%%%%%%%%%%%%%%%%%%%%%%%%%%%%%%%%%%%%%%%%%%%%%%%%%%%%%%%%%%%%%%%%%%%%%%%%%%%%%%%
%\font\nullfont=cmr10
\begin{document}

\tableofcontents


\part{Un T\'itulo largo para la parte}
\lipsum
\chapter{Un T\'itulo largo para el primer cap\'itulo}
\PartialToc
\section{AA}
\lipsum
 \begin{ejemplo}
  \lipsum[1]
  \end{ejemplo}
\section{Test section}
\lipsum
\begin{ejercicio}[Un ejemplo]
\lipsum[1]
\end{ejercicio}
\begin{problema}[Un problema]
\lipsum[1]
\end{problema}
\subsection{AAA}
\lipsum
\subsubsection{AAAA}
\lipsum
%%%%%%%%%%%%%%%%%%%%%%%%%%%%%%%%%%Figuras%%%%%%%%%%%%%%%%%%%%%%%%%%
\begin{figure}
\ffigbox[\FBwidth]
  {\includegraphics[width=.8\linewidth]{example-image}}
  {\caption{how many factors are there in a set of 12?}}
\end{figure}
%%%%%%%%%%%%%%%%%%%%%%%%%%%%%%Definicion %%%%%%%%%%%%%%%%%%%%%%%%%%
\subsubsection{AAAB}
\lipsum
\begin{defi}{Partially ordered set}{poset}
A partial order is a binary relation $\preccurlyeq$ over a set $P$ which is antisymmetric, transitive, and reflexive. A set with a partial order is called a partially ordered set (also called a poset). 
\end{defi}

\begin{lema}{Zorn's Lemma}{zorn}
Suppose a non-empty partially ordered set $P$ has the property that every non-empty chain has an upper bound in $P$. Then the set $P$ contains at least one maximal element.
\end{lema}
%\begin{teorema}{}{teo}
%\lipsum[2]
%\end{teorema}
\begin{lema}{A list test}{lsit}
\begin{itemize}
\item First.
\item Second.
\item Third.
\end{itemize}
\end{lema}

\subsection{AAB}
\lipsum
\subsection{AAC}
%\begin{nota}
%\lipsum[1]
%\end{nota}
\lipsum[4-5]
\begin{lema}{A list test}{lsit}
\begin{itemize}
\item First.
\item Second.
\item Third.
\end{itemize}
\lipsum[1-3]
\end{lema}

\chapter{Second chapter}
\PartialToc
\subsection{AAD}
\begin{MyBlock}{Some Variable Width Block}
\lipsum[4]
\end{MyBlock}

\begin{MyBlock}[.5\linewidth]{Some Title}
\lipsum[4]
\end{MyBlock}
\lipsum
\subsubsection{AADA}
\begin{solucion}
\lipsum[4]
\begin{center}
\captionof{table}{This is a table inside a \texttt{tcolorbox} environment}
\begin{tabular}{ccc}
\toprule
column1a & column2a & column3a \\
column1a & column2a & column3a \\
column1a & column2a & column3a \\
\bottomrule
\end{tabular}
\end{center}
\end{solucion}
\begin{prueba}
\lipsum
\end{prueba}
\subsubsection{AADB}
\begin{propiedades}{Propiedades}
    \includegraphics[scale=.1]{example-image-a}
    \tcblower
    Analyses to what degree the government intervenes
\end{propiedades}
\subsubsection{AADC}
\begin{propiedades}{\mbox{}}
\begin{itemize}
\item aaaa
\item bbbbb
\end{itemize}
\end{propiedades}
\vspace{20pt}
\begin{propiedad}{\mbox{}}
\begin{itemize}
\item aaaa
\item bbbbb
\end{itemize}
\end{propiedad}
\vspace{20pt}
\begin{axioma}{\mbox{}}
\begin{itemize}
\item aaaa
\item bbbbb
\end{itemize}
\end{axioma}

\lipsum
\subsubsection{AADD}
\begin{rcode}
<<my-label, eval=TRUE, dev='png'>>=
set.seed(1213)  # for reproducibility
x = cumsum(rnorm(100))
mean(x)  # mean of x
plot(x, type = 'l')  # Brownian motion
@
\end{rcode}
\begin{mybox}[listing only,listing engine=minted]
{C\'odigo de knitr}
<<my-label, eval=TRUE, dev='png'>>=
set.seed(1213)  # for reproducibility
x = cumsum(rnorm(100))
mean(x)  # mean of x
plot(x, type = 'l')  # Brownian motion
@
\end{mybox}
\bigskip
\lipsum[4]
\vspace*{20pt}
\begin{ideabox}
Did you know:\par\vskip1cm

Once upon a time there was a latch \\
that wanted to be a flip-flop?
\end{ideabox}
\vspace*{20pt}
\begin{questionbox}
Did you know:\par\vskip1cm

Once upon a time there was a latch \\
that wanted to be a flip-flop?
\end{questionbox}
\vspace*{20pt}
\lipsum
\vspace*{20pt}
\begin{apunte}
Did you know:\par\vskip1cm

Once upon a time there was a latch \\
that wanted to be a flip-flop?
\end{apunte}

\end{document}