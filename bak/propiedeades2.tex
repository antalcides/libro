\documentclass{article}

\usepackage{mwe,lipsum}
\usepackage[framemethod=tikz]{mdframed}
\usetikzlibrary{shadows}

\definecolor{mygreen}{rgb}{0.61,0.73,0.35}

\mdfdefinestyle{mystyle}{%
    innerlinewidth = 0.5pt,
    outerlinewidth = 2pt,
    linecolor = mygreen,
    tikzsetting = {draw = white, line width = 0.5pt},
    roundcorner=15pt,
    linecolor=mygreen,
    linewidth=2pt,
    topline=true,
    frametitleaboveskip=1.5\dimexpr-\ht\strutbox\relax,
}

\newenvironment{definition}[1][]{%
\ifstrempty{#1}%
    {\mdfsetup{%
    style = mystyle,
    }}%
    {\mdfsetup{%
    style = mystyle,
    frametitle={%
    \tikz{[baseline=(current bounding box.east),outer sep=0pt]
    \node[draw = white, line width = 2pt, text = white, anchor=east,rectangle,
    fill=mygreen, rounded corners, drop shadow]
    {\strut #1};
    }}}}%
    \begin{mdframed}[]\relax%
}{\end{mdframed}}

\begin{document}
\lipsum[1]
\begin{definition}[Political Factors]
\begin{minipage}{0.25\linewidth}
\includegraphics[width = \linewidth]{example-image-a}
\end{minipage}%
\hfill
\begin{minipage}{0.7\linewidth}
Analyses to what degree the government intervenes in the
economy. It includes regulations and legal issues and defines
both formal and informal rules under which the firm must
operate. Political factors include: tax policy, employment laws,
environmental regulations, trade restriction tariffs and political
stability.
\end{minipage}%
\end{definition}
\lipsum[2]
\end{document}