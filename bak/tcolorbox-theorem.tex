%this is created by Mohcine 
\documentclass[10pt,a4paper]{report}
\usepackage[margin=1in]{geometry}
\usepackage{amsthm,amssymb,amsfonts}
\usepackage{tikz,lipsum,lmodern}
\usepackage[most]{tcolorbox}
\usepackage[utf8x]{inputenc}
\newcounter{mbo}
\newcounter{mbt}
\newcounter{mbth}

\newtcolorbox{Box1}[2][]{
            before title={\stepcounter{mbo}},
                lower separated=false,
                colback=white!80!gray,
colframe=white, fonttitle=\bfseries,
colbacktitle=white!50!gray,
coltitle=black,
enhanced,
attach boxed title to top left={xshift=0.5cm,yshift=-2mm},
title={\thembo~#2},#1}


\newtcolorbox{Box2}[2][]{
            before title={\stepcounter{mbt}},
                lower separated=false,
                colback=white,
colframe=black,fonttitle=\bfseries,
colbacktitle=black,
coltitle=white,
enhanced,
attach boxed title to top left={yshift=-0.1in,xshift=0.15in},
                 boxed title style={boxrule=0pt,colframe=white,},
title={\thembt~#2},#1}

\newtcolorbox{Box3}[2][]{
            before title={\stepcounter{mbth}},
                lower separated=false,
                colback=white!80!gray,
colframe=white!20!black,fonttitle=\bfseries,
colbacktitle=white!30!gray,
coltitle=black,
enhanced,
attach boxed title to top left={xshift=0.5cm,
        yshift=-2mm},
title={\thembth~#2},#1}

\newtcolorbox{Box4}[2][]{arc=0mm,
                lower separated=false,
                colback=white!30!gray,
colframe=white!20!black,fonttitle=\bfseries,
colbacktitle=white!30!gray,
coltitle=black,
enhanced,
attach boxed title to top left={xshift=0.5cm,
        yshift=-2mm},
title=#2,#1}

\begin{document}

\begin{Box1}{Définition}
  Soit $X : \Omega \longrightarrow E$ une variable aléatoire. On appelle loi de $X$ l'application
    \[ P_{X} : X\left( \Omega \right)\to  [0,1],\quad x\mapsto \mathbb{P}\left( X=x\right)\]
\end{Box1}

\begin{Box2}{Proposition: Loi image}
 Soit $X : \Omega \longrightarrow E$  une variable aléatoire et $f : E \longrightarrow F$ . La loi de la variable aléatoire $Y=f\circ  X$
est donnée par
\[\forall y \in f\left(X\left(\Omega \right) \right),\quad \mathbb{P}\left(Y=y\right)=\sum_{x\in f^{-1}\left(\{y\} \right)}\mathbb{P}\left(X=x \right).  \]
\end{Box2}

\begin{Box4}{}
\begin{center}
{\Huge\textbf{Les méthodes à maitriser}}
\end{center}
\end{Box4}

\begin{Box3}{Méthode 26.1 : Savoir calculer la loi d'une variable aléatoire}\label{box3:1}
Il faut tout d'abord déterminer l'ensemble A des valeurs possibles de $X$, puis de calculer les
$\mathbb{P}\left(X=x\right)$, $x$ parcourant l'ensemble des valeurs prises par $X$. On représente généralement le résultat sous forme de tableau
\end{Box3}
eejmplo\ref{box3:1}
\end{document}